\begin{problem}
Определение функции, сюръекции, инъекции и биекции.
\end{problem}

\begin{definition}
	Декартовым произведением $X\times Y$
	множеств $X$ и $Y$ называют множество
	всевозможных пар $(x, y),$ где первый
	элемент $x$ каждой пары принадлежит $X$,
	а второй ее элемент $y$ принадлежит $Y$.
\end{definition}
\begin{definition}
	Функцией $f,$ определённой на множестве $X$ и
	принимающей значения во множестве $Y,$
	называется подмножество декартова
	произведения $X\times Y,$ если
	выполнено следующее условие:
	$\forall x\in X\;\exists! \textrm{ пара }
		(x, y)\in f$. При этом пишут $y=f(x)$.
	Элемент $y$ называют образом $x$,
	элемент $x$ -- прообразом элемента
	$y$, для функции принято обозначение
	$f: X\rightarrow Y$.

	Множество $f(X)$ всех элементов $f(x)\in
		Y$ называется образом множества $X$.
	Короче, это записывается так:
	$$
		f(X)=\{y\in Y|y=f(x), \;x\in X\},
	$$
	а само $X$ называется прообразом
	множества $f(X)$.
\end{definition}

\begin{definition}
	1) Функция $f: X\rightarrow Y$
	называется сюръекцией
	(накрытием), если для всякого
	$y\in Y$ существует такое $x\in X$,
	что $y=f(x)$.\\
	2) Функция $f: X\rightarrow Y$
	называется инъекцией
	(вложением), если из равенства
	$f(x)=f(y)$ следует, что $x=y$.\\
	3) Функция, являющаяся одновременно
	сюръекцией и инъекцией, называется
	биекцией или взаимно-однозначным
	отображением.
\end{definition}

\newpage
\begin{problem}
Определение того, что множество $A$ лежит левее множества $B$, разделяющего элемента
и формулировка принципа полноты.
\end{problem}

\begin{definition}
	Говорят, что
	множество $A$ лежит левее множества
	$B,$ если $a\leq b$ для всяких $a\in A$
	и $b\in B$.
\end{definition}

\begin{definition}
	Пусть
	множество $A$ лежит левее $B$.
	Тогда говорят, что число $c$
	разделяет множества $A$ и $B,$
	если $a\leq c$ и $c\leq b$
	для всех $a\in A, b\in B$.
\end{definition}

\begin{definition}
	Говорят,
	что для числового множества
	выполняется принцип полноты, если
	для любых двух его подмножеств,
	одно из которых лежит левее
	другого, найдётся элемент,
	разделяющий эти множества.
\end{definition}

\newpage
\begin{problem}
Определение множества вещественных чисел (с помощью 11 свойств) и формулировка
принципа полноты для множества вещественных чисел.
\end{problem}

Множество действительных чисел,
которое мы обозначим $\R,$
должно удовлетворять ряду условий.
Во-первых, сумма и произведение любых элементов
этого множества снова является элементом
этого множества.
Кроме того, выполнены следующие свойства.

1) Для всех $a, \ b, \ c \in\R$
выполнено условие \emph{ассоциативности по сложению}:
$$
	a+(b+c)=(a+b)+c.
$$

2) Для всех $a, \ b \in\R$
выполнено условие \emph{коммутативности по сложению}:
$$
	a+b=b+a.
$$

3) Существует \emph{нейтральный элемент относительно
	операции сложения}
$0 \ \in\R,$ такой, что для всех $a \in\R$
$a+0=a$.

4) Для всякого $a \in\R$ найдётся
\emph{противоположный элемент} $b \ \in\R,$
такой, что $a+b=0$ (его обычно обозначают через
$-a$).

5) Для всех $a, \ b, \ c \in\R$
выполнено условие \emph{ассоциативности по умножению}:
$$
	a\cdot(b\cdot c)=(a\cdot b)\cdot c.
$$

6) Для всех $a, \ b, \ c \in\R$
выполнено условие \emph{коммутативности по умножению}:
$$
	a\cdot b=b\cdot a.
$$

7) Существует \emph{нейтральный элемент относительно
	операции умножения}
$1\in\R,$ такой, что для всех $a\in\R$
$a\cdot1=a$.

8) Для всякого $a \ \in\R\setminus\{0\}$ найдётся
\emph{обратный элемент} $b \ \in\R\setminus\{0\},$
такой, что $a\cdot b=1$ (его обычно обозначают через
$a^{-1}$).

Отметим, что эти свойства означают, что
множество действительных чисел
является \emph{абелевой группой по сложению},
а множество всех действительных чисел
без нуля является \emph{абелевой группой по умножению}.

9) Для всех $a, \ b, \ c \in\R$
выполнено условие \emph{дистрибутивности}:
$$
	a\cdot(b+c)=a\cdot b+a\cdot c.
$$

Множества, на котором выполнены эти
девять свойств, называется \emph{полем}.
Помимо этих девяти свойств есть ещё два.

10) Для всех $a, \ b \in\R$
либо $a\leq b,$ либо $b\leq a,$
то есть любые два элемента $\R$
можно сравнить (множество, где сравнимы
любые два элемента, называется
\emph{линейно упорядоченным}). При
этом выполнены два свойства:\\
а) для всех $a, \ b, \ c \in\R,$ таких,
что $a\leq b$ выполнено $a+c\leq b+c;$\\
б) для всех $a, \ b \in\R$ и
$c \in\R, \ c\geq0$ таких,
что $a\leq b$ выполнено $a\cdot c\leq b\cdot c$.

11) На множестве вещественных чисел
выполнен принцип полноты.

\newpage
\begin{problem}
Аксиома Архимеда и следствие из аксиомы Архимеда (Лемма 1, Лекция 1).
\end{problem}

\begin{axiom} \textbf{(Аксиома Архимеда).}
	Для любого положительного вещественного числа
	$a$ существует такое натуральное число $n,$ что
	$na\geq1$ (с помощью кванторов: $\forall\;a\in\R\;
		\wedge a>0\exists\;n\in\N:\;na\geq1$).
\end{axiom}
Из аксиомы Архимеда вытекает полезное
следствие.

\begin{lemma}
	1) $\forall x,\;y\in\R:\;x<y\;\exists\;
		m/n\in\Q:\;x<\frac{m}{n}<y;$
	2) $\forall x,\;y\in\R:\;x<y\;\exists\;
		c\in\mathbb{I}:\\x<c<y.$
\end{lemma}
Более просто свойство 1) формулируется так:
между любыми двумя различными действительными
числами лежит рациональное число. В свойстве
2) речь идёт об иррациональном числе.

\newpage
\begin{problem}
Система и последовательность вложенных и стягивающихся отрезков и лемма о вложенных отрезках.
\end{problem}


\begin{definition}
	1) Системой вложенных отрезков называется
	множество $M$, состоящее из отрезков,
	в котором для любых $I_1,\;I_2\in M$
	выполнено либо условие $I_1\subset I_2,$
	либо условие $I_2\subset I_1.$\\
	2) Если при этом в $M$
	все отрезки
	занумерованы,
	и любой отрезок с большим
	номером содержится в любом отрезке
	с меньшим номером, то множество $M$
	называют \textbf{последовательностью
		вложенных отрезков.}
\end{definition}

\begin{definition}
	Последовательность вложенных отрезков называется
	стягивающейся, если для любого
	числа $\varepsilon>0$ в этой последовательности
	найдётся отрезок, длина которого меньше $\varepsilon.$
\end{definition}

\begin{theorem} \textbf{\textrm{(Лемма о вложенных отрезках)}}
	1) Пусть дана система $M$ вложенных отрезков. Тогда
	существует такое число $c\in\R,$ что для любого отрезка
	$I\in M$ имеем $c\in I,$ то есть все отрезки
	множества $M$ имеют общий элемент $c$.\\
	2) Если множество $M$
	является последовательностью стягивающихся отрезков,
	то элемент $c$ единственен.
\end{theorem}


\newpage
\begin{problem}
Определение числовой последовательности и определение предела последовательности
(оба варианта, то есть определения 1 и 2, Лекция 3).
\end{problem}

\begin{definition}
	Функция, $f:\N\rightarrow\R,$
	определённая на множестве натуральных
	чисел и принимающая значения во множестве
	действительных чисел, называется числовой
	последовательностью.
	Значения $f(n)$ функции $f$ обозначают $a_n.$
	Последовательность также часто обозначают
	$\{a\}_{n=1}^{\infty},$ отождествляя её с
	множеством её значений.
\end{definition}
\begin{definition}
	Число $A$ называется пределом
	последовательности $\{a_n\}_{n=1}^{\infty},$
	если для любой окрестности точки $A$
	существует такое натуральное число $N,$
	что при всех натуральных $n>N$ числа $a_n$ лежат в этой
	окрестности. На языке кванторов это определение
	запишется так:
	$$
		\lim\limits_{n\rightarrow\infty}a_n=A\Leftrightarrow
		\forall U(A)\exists N\in\mathbb{N}: \forall n>N\; a_n\in U(A).
	$$
\end{definition}

\begin{definition}
	Число $A$ называется пределом
	последовательности $\{a_n\}_{n=1}^{\infty},$
	если для любого положительного числа $\varepsilon$
	существует такое натуральное число $N,$
	что при всех $n>N$ выполнено неравенство
	$|a_n-A|<\varepsilon.$
	На языке кванторов это определение
	запишется так:
	$$
		\lim\limits_{n\rightarrow\infty}a_n=A\Leftrightarrow
		\forall \varepsilon>0 \ \exists N\in\mathbb{N}:
		\forall n>N \ |a_n-A|<\varepsilon.
	$$
\end{definition}

\newpage
\begin{problem}
Единственность предела последовательности, определение ограниченной последовательности, ограниченность сходящейся последовательности и лемма об отделимости.
\end{problem}

\begin{lemma}
	Пусть у последовательности $\{a_n\}_{n=1}^{\infty}$
	существует предел. Тогда этот предел единственен.
\end{lemma}


\begin{definition}
	Последовательность $\{a_n\}_{n=1}^{+\infty}$
	называется ограниченной, если существуют
	такие числа $c, C\in\mathbb{R},$ что
	$c\leq a_n\leq C$ при всех $n\in\mathbb{N}.$
	Равносильным определением будет следующее:
	последовательность $\{a_n\}_{n=1}^{+\infty}$
	называется ограниченной, если существует
	такое число $M>0,$ что
	$|a_n|\leq M$ при всех $n\in\mathbb{N}.$
\end{definition}
\begin{lemma}
	Сходящаяся последовательность ограничена.
\end{lemma}

\begin{lemma}
	(\textbf{Лемма об отделимости}).
	Пусть $\lim\limits_{n\rightarrow\infty}a_n=A$
	и $A\neq0.$ Тогда существует такое
	натуральное число $N,$ что для всех
	$n>N$ выполнено неравенство
	$|a_n|>\frac{|A|}{2}.$
\end{lemma}

\newpage
\begin{problem}
Арифметика пределов, переход к пределу в неравенстве и лемма о зажатом пределе.
\end{problem}

\begin{theorem}(\textbf{Арифметика пределов}).
	Пусть $\lim\limits_{n\rightarrow\infty}a_n=A,$
	$\lim\limits_{n\rightarrow\infty}b_n=B.$ Тогда:\\
	1) $\forall\alpha, \beta\in\mathbb{R}\;\lim\limits_{n\rightarrow\infty}(\alpha a_n
		+\beta b_n)=\alpha\lim\limits_{n\rightarrow\infty}
		a_n+\beta\lim\limits_{n\rightarrow\infty}b_n=
		\alpha A+\beta B;$\\
	2) $\lim\limits_{n\rightarrow\infty}(a_n
		\cdot b_n)=\lim\limits_{n\rightarrow\infty}
		a_n\cdot \lim\limits_{n\rightarrow\infty}b_n=
		A\cdot B;$\\
	3) если $B\neq0,$ $b_n\neq0$ при всех
	натуральных $n,$ то
	$\lim\limits_{n\rightarrow\infty}\frac{a_n}{b_n}
		=\frac{\lim\limits_{n\rightarrow\infty}
			a_n}{\lim\limits_{n\rightarrow\infty}b_n}=
		\frac{A}{B}.$
\end{theorem}

\begin{lemma}
	(\textbf{Переход к пределу в неравенстве}).
	Пусть $$\lim\limits_{n\rightarrow\infty}a_n=A
		\textrm{ и} \lim\limits_{n\rightarrow\infty}b_n=B,$$
	а также существует такое натуральное $n_0,$
	что $a_n\leq b_n$ при всех $n>n_0.$
	Тогда $A\leq B.$
\end{lemma}

\begin{theorem} (\textbf{Лемма о зажатом пределе}).
	Пусть
	$\lim\limits_{n\rightarrow\infty}a_n=
		\lim\limits_{n\rightarrow\infty}b_n=A$
	и $a_n\leq c_n\leq b_n,$ начиная
	с некоторого натурального $n.$
	Тогда $\lim\limits_{n\rightarrow\infty}c_n=A.$
\end{theorem}

\newpage
\begin{problem}
Определение бесконечно малой последовательности, свойство произведения бесконечно
малой последовательности на ограниченную и определение предела в терминах бесконечно малых последовательностей.
\end{problem}

\begin{definition}
	Последовательность $\{a\}_{n=1}^{\infty}$
	бесконечно малая, если
	$\lim\limits_{n\rightarrow\infty}a_n=0.$
\end{definition}
\begin{lemma}
	Если
	последовательность $\{a_n\}_{n=1}^{\infty}$
	бесконечно малая, а последовательность
	$\{b_n\}_{n=1}^{\infty}$ ограниченная, то
	последовательность $\{a_n\cdot
		b_n\}_{n=1}^{\infty}$ бесконечно малая.
\end{lemma}

\begin{definition}
	Число $A$ называется пределом
	последовательности $\{a_n\}_{n=1}^{\infty},$
	если существует такая бесконечно малая
	последовательность $\{\alpha_n\}_{n=1}^{\infty},$
	что $\forall n\in\N\;a_n=A+\alpha_n.$
\end{definition}

\newpage
\begin{problem}
Определение верхних и нижних граней, а также два определения точной верхних и
нижних граней (Определения 1 и 2, Лекция 4). Существование точной верхней и нижней
граней ограниченного множества.
\end{problem}

\begin{definition}
	Пусть дано непустое подмножество $A$ множества
	действительных чисел. Число $C$ называется
	верхней гранью множества $A,$ если
	$a\leq C$ при всех $a\in A.$ Если множество
	$A$ имеет хотя бы одну верхнюю грань, то оно
	называется ограниченным сверху.
	Наименьшая из верхних граней множества
	$A$ (если она существует) называется его точной верхней гранью
	и обозначается $\sup A$ (читается: супремум.)

	Число $c$ называется
	нижней гранью множества $A,$ если
	$a\geq c$ при всех $a\in A.$ Если множество
	$A$ имеет хотя бы одну нижнюю грань, то оно
	называется ограниченным снизу.
	Наибольшая из нижних граней множества
	$A$ (если существует) называется его точной нижней гранью
	и обозначается $\inf A$ (читается: инфимум.)

	Множество, ограниченное сверху и снизу,
	называется ограниченным.
\end{definition}
\begin{definition}
	Число
	$C$ называется точной верхней гранью
	множества $A,$ если:\\1) $a\leq C$ для всех
	$a\in A;$\\ 2) $\forall\varepsilon>0
		\textrm{ }\exists a\in A: a>C-\varepsilon.$
\end{definition}

\begin{theorem}
	Пусть множество $A$ непусто и ограничено
	сверху. Тогда существует $\sup A.$
\end{theorem}

\newpage
\begin{problem} Определение монотонной последовательности. Теорема Вейерштрасса для последовательностей. Определение числа $e$.
\end{problem}

\begin{definition}
	Последовательность $\{a_n\}_{n=1}^{+\infty}$
	называется неубывающей, если $a_n\leq a_{n+1}$
	при всех $n\in\mathbb{N}$ и невозрастающей,
	если $a_n\geq a_{n+1}$ при всех $n\in\mathbb{N}.$
	Невозрастающая или неубывающая последовательность
	называется \textbf{монотонной.}

	Последовательность $\{a_n\}_{n=1}^{+\infty}$
	называется возрастающей, если $a_n<a_{n+1}$
	при всех $n\in\mathbb{N}$ и убывающей,
	если $a_n>a_{n+1}$ при всех $n\in\mathbb{N}.$
	Возрастающая или убывающая последовательность
	называется \textbf{ строго монотонной.}
\end{definition}
\begin{theorem} \textrm{(Вейерштрасс)}.
	Монотонная и ограниченная последовательность
	имеет предел.
\end{theorem}

\begin{theorem}
	Последовательность $a_n=\left(1+\frac{1}{n}\right)^n$
	имеет предел.
\end{theorem}
\begin{definition}
	Числом $e$ называют предел последовательности
	$a_n=\left(1+\frac{1}{n}\right)^n,$
	то есть по определению полагают
	$e=\lim\limits_{n\rightarrow\infty}
		\left(1+\frac{1}{n}\right)^n.$
\end{definition}


\newpage
\begin{problem}
Определение того, что $\lim\limits_{n \to + \infty} a_n = +\infty$. Теорема о сходимости к e последовательностей
более общего вида (Предложение 1, Лекция 5).
\end{problem}

\begin{definition}
	Говорят, что
	$\lim\limits_{n\rightarrow\infty}a_n=+\infty,$
	если
	$$
		\forall M\;\exists N:\;\forall n>N\;
		a_n>M.
	$$
\end{definition}
\begin{theorem}
	Пусть заданы последовательности
	$\{p_n\}_{n=1}^{+\infty}$ и
	$\{q_n\}_{n=1}^{+\infty},$ причём
	$$\lim\limits_{n\rightarrow\infty}p_n=+\infty
		\textrm{ и }\lim\limits_{n\rightarrow\infty}q_n=-\infty.$$
	Тогда
	$$
		\lim\limits_{n\rightarrow\infty}\left(1+\frac{1}{p_n}\right)^{p_n}=e
		\textrm{ и }
		\lim\limits_{n\rightarrow\infty}\left(1+\frac{1}{q_n}\right)^{q_n}=e.
	$$
\end{theorem}

\newpage
\begin{problem}
Определение подпоследовательности и частичного предела. Частичные пределы последовательности, имеющей предел (Предложение 2, Лекция 5). Теорема Больцано – Вейерштрасса.\end{problem}
\begin{definition}
	Пусть задана последовательность
	$\{a_n\}_{n=1}^{\infty}$ и возрастающая
	последовательность натуральных чисел
	$n_1<n_2<n_3<...<n_m<....$ Возьмём
	элементы последовательности
	$\{a_n\}_{n=1}^{\infty}$ с номерами
	$n_1<n_2<n_3<...<n_m<....$ Мы снова получим
	последовательность $b_k=a_{n_k},$ которая
	называется \textbf{подпоследовательностью}
	последовательности $\{a_n\}_{n=1}^{\infty}.$

	Число $a\in\mathbb{R}$ называется
	\textbf{частичным пределом} последовательности
	$\{a_n\}_{n=1}^{\infty},$ если найдётся
	подпоследовательность $\{a_{n_k}\}_{k=1}^{\infty}$
	последовательности $\{a_n\}_{n=1}^{\infty},$
	для которой число $a$ является пределом,
	то есть $\lim\limits_{k\rightarrow\infty}
		a_{n_k}=a.$
\end{definition}

\begin{lemma}
	Если последовательность имеет предел,
	то любая её подпоследовательность сходится
	к тому же пределу.
\end{lemma}

\begin{theorem}\textbf{(Больцано -- Вейерштрасс.)}
	Из всякой ограниченной последовательности
	можно выбрать сходящуюся подпоследовательность.
\end{theorem}


\newpage
\begin{problem}
Определение верхнего и нижнего предела последовательности. Структура множества
частичных пределов (Теорема 3, Лекция 5). Критерий существования предела в терминах
частичных пределов (Теорема 4, Лекция 5).
\end{problem}

Пусть последовательность
$\{a_n\}_{n=1}^{\infty}$ ограничена.
Рассмотрим последовательность $$M_n=
	\sup\limits_{k>n}a_k.$$ С увеличением
$n$ точная верхняя грань не может
увеличиться, так как супремум
множества $\{a_{n+1}, \ a_{n+2}, \ ...\},$
равный $M_n$ нее меньше, чем супремум
множества $\{a_{n+2}, \ a_{n+3}, \ ...\},$
который равен $M_{n+1}$.
Таким образом, последовательность
$\{M_n\}_{n=1}^{\infty}$ не возрастает.
Кроме того,
$M_n\geq a_k$ при всех натуральных $k>n,$
что в силу ограниченности последовательности
$\{a_n\}_{n=1}^{\infty}$  означает, что
последовательность $\{M_n\}_{n=1}^{\infty}$
ограничена снизу. Следовательно по теореме
Вейерштрасса последовательность
$\{M_n\}_{n=1}^{\infty}$
имеет предел. Аналогично
доказывается, что последовательность
$m_n=\inf\limits_{k>n}a_k$ имеет предел.
Пусть $\lim\limits_{n\rightarrow\infty}
	M_n=M,$ а
$\lim\limits_{n\rightarrow\infty}m_n=m.$
\begin{definition}
	Пусть последовательность
	$\{a_n\}_{n=1}^{\infty}$ ограничена.
	Тогда число $M$ называют \textbf{верхним
		пределом} последовательности
	$\{a_n\}_{n=1}^{\infty},$ а число
	$m$ -- \textbf{нижним пределом}
	этой последовательности.
	Соответствующие обозначения:
	$
		M:=\uplim\limits_{n\rightarrow\infty}a_n,\;
		m:=\lowlim\limits_{n\rightarrow\infty}a_n.
	$
\end{definition}
\begin{theorem}
	Если последовательность $\{a_n\}_{n=1}^{\infty}$
	ограничена, то $\uplim\limits_{n\rightarrow\infty}a_n$
	и $\lowlim\limits_{n\rightarrow\infty}a_n$ являются
	частичными пределами этой последовательности
	и все частичные пределы
	последовательности $\{a_n\}_{n=1}^{\infty}$
	принадлежат отрезку
	$[\lowlim\limits_{n\rightarrow\infty}a_n,
				\uplim\limits_{n\rightarrow\infty}a_n].$
\end{theorem}


\begin{theorem}
	Ограниченная последовательность имеет предел
	тогда и только тогда, когда у неё только
	один частичный предел.
\end{theorem}

\newpage
\begin{problem}
Определение фундаментальной последовательности. Формулировка критерия Коши.
\end{problem}
\begin{definition}
	Последовательность
	$\{a_n\}_{n=1}^{\infty}$
	называется фундаментальной
	(или последовательностью Коши),
	если для всякого числа $\varepsilon>0$
	найдётся такое число
	$N\in\N,$ что при всех
	$n,\;m>N$ выполняется неравенство
	$|a_n-a_m|<\varepsilon.$
	Более коротко:
	$$
		\forall\varepsilon>0\;\exists
		N\in\mathbb{N}:\;\forall
		n,\;m>N\;
		|a_n-a_m|<\varepsilon.
	$$
\end{definition}

\begin{theorem} \textbf{(Критерий Коши.)}
	Последовательность $\{a_n\}_{n=1}^{\infty}$
	имеет предел \textbf{тогда и только
		тогда}, когда она фундаментальна.
\end{theorem}


\newpage
\begin{problem}
Определение числового ряда, частичной суммы, последовательности частичных сумм
и суммы ряда. Критерий Коши сходимости ряда. Необходимый признак сходимости ряда.
\end{problem}

\begin{definition}
	Если задана числовая последовательность
	$\{a_n\}_{n=1}^{\infty},$ то сумма
	$$a_1+a_2+a_3+...=\sum\limits_{n=1}\limits^{\infty}a_n$$
	называется \textbf{числовым рядом}.
	Сумма $S_n:=a_1+a_2+...+a_n$
	называется $n-$ой \textbf{частичной суммой} ряда.
	Последовательность $\{S_n\}_{n=1}^{\infty}$
	называется \textbf{последовательностью частичных
		сумм} ряда.
\end{definition}

\begin{definition}
	Если последовательность частичных
	сумм имеет предел, то есть существует такое
	действительное число $S,$ что
	$\lim\limits_{n\rightarrow\infty}S_n=S,$
	то говорят, что ряд сходится, а
	число $S$ называют \textbf{суммой ряда}.
	Если последовательность частичных сумм
	не имеет предела или имеет бесконечный
	предел, то говорят, что ряд расходится.
\end{definition}

\begin{theorem} \textbf{(Критерий Коши сходимости ряда.)}
	Ряд $\sum\limits_{n=1}\limits^{\infty}a_n$ сходится тогда
	и только тогда, когда для любого числа $\varepsilon>0$
	существует такое натуральное число $N,$ что
	при любом $n>N$ и любом $p\in\mathbb{N}$
	выполнено неравенство $|a_{n+1}+...+a_{n+p}|<\varepsilon.$
\end{theorem}

\begin{lemma}\textbf{(Необходимый признак сходимости ряда).}
	Пусть ряд $\sum\limits_{n=1}\limits^{\infty}a_n$
	сходится. Тогда $\lim\limits_{n\rightarrow\infty}
		a_n=0.$
\end{lemma}

\newpage
\begin{problem}
Абсолютная и условная сходимость ряда. Критерий сходимости рядов с неотрицательными членами.
\end{problem}

\begin{definition}
	Ряд $\sum\limits_{n=1}\limits^{\infty}a_n$
	называется \textbf{абсолютно сходящимся},
	если сходится ряд
	$\sum\limits_{n=1}\limits^{\infty}|a_n|.$
	Если ряд $\sum\limits_{n=1}\limits^{\infty}a_n$
	сходится, а ряд
	$\sum\limits_{n=1}\limits^{\infty}|a_n|$
	расходится, то ряд
	$\sum\limits_{n=1}\limits^{\infty}a_n$
	называется \textbf{условно сходящимся}.
\end{definition}

\begin{lemma}
	Если $a_n\geq0$ $\forall n\in\mathbb{N},$
	то ряд $\sum\limits_{n=1}\limits^{\infty}a_n$
	сходится тогда и только тогда,
	когда ограничена последовательность его
	частичных сумм.
\end{lemma}

\newpage
\begin{problem}
Признак сравнения. Признак сравнения в предельной форме. Мажорантный признак
Вейерштрасса. Признак разрежения Коши.
\end{problem}

\begin{theorem} \textbf{(Признак сравнения).}
	Пусть при всех натуральных $n,$
	начиная с некоторого номера,
	выполнены неравенства $0\leq a_n
		\leq b_n.$ Тогда из сходимости
	ряда $\sum\limits_{n=1}\limits^{\infty}b_n$
	следует сходимость ряда
	$\sum\limits_{n=1}\limits^{\infty}a_n,$
	а из расходимости ряда
	$\sum\limits_{n=1}\limits^{\infty}a_n$
	следует расходимость ряда
	$\sum\limits_{n=1}\limits^{\infty}b_n.$
\end{theorem}

\begin{theorem} \textbf{(Признак сравнения в предельной форме).}
	Пусть при всех натуральных $n,$
	начиная с некоторого числа $N\in\N$,
	выполнены неравенства $a_n\geq0,\;b_n>0,$
	а также
	$\lim\limits_{n\rightarrow+\infty}\frac{a_n}{b_n}=A,$
	причём $A\in(0, +\infty).$
	Тогда ряды $\sum\limits_{n=1}\limits^{\infty}b_n$
	и $\sum\limits_{n=1}\limits^{\infty}a_n$
	или оба сходятся, или оба расходятся.
\end{theorem}

\begin{lemma} \textbf{(Мажорантный признак Вейерштрасса).}
	Пусть при всех натуральных $n,$
	начиная с некоторого числа $N\in\N$,
	выполнены неравенства $b_n\geq|a_n|$
	и ряд $\sum\limits_{n=1}\limits^{\infty}b_n$
	сходится. Тогда ряд
	$\sum\limits_{n=1}\limits^{\infty}a_n$
	сходится.
\end{lemma}

\begin{theorem}\textbf{(Признак разрежения Коши).}
	Если последовательность
	$\{a_n\}_{n=1}^{\infty}$ не возрастает
	и $a_n\geq0$ при любом натуральном
	$n,$ то ряд
	$\sum\limits_{n=1}\limits^{+\infty}a_n$
	сходится тогда и только тогда, когда
	сходится ряд
	$\sum\limits_{n=1}\limits^{\infty}2^na_{2^n}.$
\end{theorem}

\newpage
\begin{problem}
Признак Даламбера. Признак Коши.
\end{problem}

\begin{theorem} \textbf{(Признак Даламбера).}
	Пусть дан ряд
	$\sum\limits_{n=1}\limits^{\infty}a_n$
	и $\lim\limits_{n\rightarrow+\infty}
		\left|\frac{a_{n+1}}{a_n}\right|=q.$
	Тогда:\\
	1) если $q<1,$ то ряд
	$\sum\limits_{n=1}\limits^{\infty}a_n$
	абсолютно сходится;\\
	2) если $q>1,$ то ряд
	$\sum\limits_{n=1}\limits^{\infty}a_n$
	расходится;\\
	3) если $q=1,$ то ряд
	$\sum\limits_{n=1}\limits^{\infty}a_n$
	может как абсолютно сходиться, так и
	расходиться ( в том смысле, что он
	не сходится даже условно).
\end{theorem}

\begin{theorem} \textbf{(Радикальный признак Коши).}
	Пусть дан ряд
	$\sum\limits_{n=1}\limits^{\infty}a_n$
	и $\uplim\limits_{n\rightarrow\infty}
		\sqrt[n]{|a_n|}=q.$
	Тогда:\\
	1) если $q<1,$ то ряд
	$\sum\limits_{n=1}\limits^{\infty}a_n$
	абсолютно сходится;\\
	2) если $q>1,$ то ряд
	$\sum\limits_{n=1}\limits^{\infty}a_n$
	расходится;\\
	3) если $q=1,$ то ряд
	$\sum\limits_{n=1}\limits^{\infty}a_n$
	может как абсолютно сходиться, так и расходиться.
\end{theorem}

\newpage
\begin{problem}
Определение покрытия. Принцип Бореля – Лебега. Предельная точка множества (оба
определения). Принцип Больцано – Вейерштрасса.
\end{problem}
\begin{definition}
	Система множеств $S=\{X\}$ называется
	покрытием множества $Y,$ если
	$Y\subseteq\bigcup\limits_{X\in S}X.$
\end{definition}

\begin{theorem} (\textbf{Принцип Бореля -- Лебега.})
	Пусть система интервалов $S=\{J\}$ является покрытием
	отрезка $I=[a, b].$ Тогда из системы $S$ можно
	выбрать конечную подсистему, также являющуюся
	покрытием $I.$
\end{theorem}

\begin{definition}
	Точка $a$ называется предельной
	точкой множества $X,$ если в любой
	окрестности точки $a$ содержится
	бесконечно много элементов множества
	$X.$
\end{definition}

\begin{definition}
	Точка $a$ называется предельной
	точкой множества $X,$ если в любой
	проколотой окрестности точки $a$ содержится
	хотя бы один элемент множества
	$X.$
\end{definition}

\begin{theorem} (\textbf{Принцип Больцано -- Вейерштрасса.})
	Пусть множество $X$ бесконечно и ограничено. Тогда оно
	имеет хотя бы одну предельную точку.
\end{theorem}


\newpage
\begin{problem}
Внутренняя, граничная и изолированная точка. Открытое и замкнутое множество.
Критерий замкнутости множества (Предложение 2, лекция 9).
\end{problem}

\begin{definition}
	Пусть $X$ -- непустое множество. Точка $a$
	множества $X$ называется \textbf{внутренней
		точкой} $X$,
	если существует такая её окрестность $U(a),$
	что $U(a)$ содержится во множестве $X.$

	Точка $b$ называется
	\textbf{граничной точкой множества} $X$,
	если в любой её окрестности $U(b)$ содержатся как
	точки, принадлежащие множеству
	$X,$ так и точки, не
	принадлежащие этому множеству.

	Точка $c$ множества $X$ называется
	\textbf{изолированной точкой} $X$,
	если найдётся такая её окрестности $U(c),$
	что в ней нет других точек из $X,$ кроме $c.$
\end{definition}

\begin{definition}
	Множество $X$ в $\R$ называется \textbf{открытым},
	если оно состоит только из внутренних
	точек.

	Множество $Y$ в $\R$ называется \textbf{замкнутым},
	если дополнение к нему до $\R$ открыто.
\end{definition}

\begin{lemma}
	Множество $Y$ является замкнутым тогда
	и только тогда, когда оно содержит
	все свои предельные точки.
\end{lemma}


\begin{lemma}
	Пусть $A$ -- множество всех частичных
	пределов последовательности
	$\{a_n\}_{n=1}^{+\infty}.$ Тогда
	$A$ -- замкнутое множество.
\end{lemma}
\newpage
\begin{problem}
Определение предела по Коши (в том числе через окрестности). Определение предела
при $x \to +\infty$. Определение предела по Гейне. Эквивалентность определений по Коши и
Гейне.
\end{problem}

\begin{definition}
	(\textbf{Определение предела по Коши}).
	Пусть функция $f$ определена на множестве
	$E\subset\R$ и пусть $a$ -- предельная
	точка множества $E.$ Число $A$ называется
	пределом функции $f$ в точке $a$ по множеству
	$E,$ если для любого $\varepsilon>0$
	существует такое $\delta>0,$ что
	для любого такого $x\in E,$ что
	$0<|x-a|<\delta$
	выполняется неравенство $|f(x)-A|<\varepsilon.$
	Запись с помощью кванторов:
	$$
		\lim\limits_{x\rightarrow a}f(x)=A\Leftrightarrow
		\forall\varepsilon>0\;\exists\delta>0:\;
		\forall x \in E\wedge\;0<|x-a|<\delta\;
		|f(x)-A|<\varepsilon.
	$$
\end{definition}

\begin{definition} (\textbf{Определение предела через окрестности}).
	Число $A$
	называется пределом функции $f$
	в точке $a$ по множеству $E,$ если
	для любой $\varepsilon$-окрестности
	$V_{\varepsilon}(A)$ точки $A$ существует
	такая проколотая $\delta$-окрестность
	$\stackrel{\circ}{U}_{\delta}(a)$ точки $a,$
	что для любого $x \in E\cap
		\stackrel{\circ}{U}_{\delta}(a)$ выполнено
	$f(x)\in V_{\varepsilon}(A),$ то есть
	выполняется неравенство $|f(x)-A|<\varepsilon.$
	Запись с помощью кванторов:
	$$
		\lim
		\limits_{x\rightarrow a}f(x)=A\Leftrightarrow
		\forall\;V_{\varepsilon}(A)\;\exists\;
		\stackrel{\circ}{U}_{\delta}(a):
		\forall x \in E\cap\stackrel{\circ}{U}_{\delta}(a)\;
		f(x)\in\;V_{\varepsilon}(A).
	$$
	Итак, для любой $\varepsilon$-окрестности точки $A$
	найдётся проколотая $\delta$-окрестность точки
	$a,$ образ пересечения которой со множеством
	$E$ при функции $f$ содержится в
	$\varepsilon$-окрестности точки $A.$
\end{definition}

\begin{definition}
	Пусть функция $f$ определена на множестве
	$E\subset\R.$ Число $A$ называется
	пределом функции $f$ при
	$x\rightarrow+\infty$ по множеству
	$E,$ если для любого $\varepsilon>0$
	существует такое $\delta>0,$ что
	для любого такого $x\in E,$ что
	$x>\delta$
	выполняется неравенство $|f(x)-A|<\varepsilon.$
	Запись с помощью кванторов:
	$$
		\lim\limits_{x\rightarrow +\infty}f(x)=A\Leftrightarrow
		\forall\varepsilon>0\;\exists\delta>0:\;
		\forall x \in E\wedge\;x>\delta\;
		|f(x)-A|<\varepsilon.
	$$
\end{definition}

\begin{definition}
	(\textbf{Определение предела по Гейне}).
	Пусть функция $f$ определена на множестве
	$E\subset\R$ и пусть $a$ -- предельная
	точка множества $E.$ Число $A$ называется
	пределом функции $f$ в точке $a$ по множеству
	$E,$ если для любой последовательности
	$\{a_n\}_{n=1}^{\infty},$ такой, что
	$a_n \in E, a_n\neq a$
	$\forall n \in \N,$ $a_n\rightarrow a$
	при $n\rightarrow\infty,$
	выполняется равенство $\lim
		\limits_{n\rightarrow\infty}f(a_n)=A.$
	Запись с помощью кванторов:
	$$
		\lim\limits_{x\rightarrow a}f(x)=A\;\Leftrightarrow\;
		\forall \{a_n\}_{n=1}^{\infty}:\;
		a_n \in E\setminus\{a\}\;\forall n \in \N\wedge
		\lim\limits_{n\rightarrow\infty}a_n=a\;
		\lim\limits_{n\rightarrow\infty}f(a_n)=A.
	$$
\end{definition}

\begin{theorem}
	$\lim\limits_{x\rightarrow a}f(x)=A$ в смысле Коши
	$\Leftrightarrow$
	$\lim\limits_{x\rightarrow a}f(x)=A$ в смысле Гейне.
\end{theorem}

\newpage
\begin{problem}
Свойства предела (Теорема 2, Лекция 10).
\end{problem}

\begin{theorem} Пусть функции $f,$ $g$ и $h$
	определены на некотором множестве $E\subset\R,$
	$a$ -- предельная точка множества $E.$
	Пусть $\lim\limits_{x\rightarrow a}f(x)=A,$ а
	$\lim\limits_{x\rightarrow a}g(x)=B$. Тогда:\\
	\textbf{1)} $A$ -- единственный предел функции $f$
	(\textbf{единственность предела});\\
	\textbf{2)} $\lim\limits_{x\rightarrow a}(f(x)\pm g(x))=A\pm B;$\\
	\textbf{3)} $\lim\limits_{x\rightarrow a}(f(x)\cdot g(x))=A\cdot B;$\\
	\textbf{4)} $\lim\limits_{x\rightarrow a}(f(x)/g(x))=A/B$
	($g(x)\neq0$ $\forall x \in E,$ $B\neq0$)
	(\textbf{арифметика предела});\\
	\textbf{5)} если $f(x)\leq g(x)$ в пересечении некоторой
	проколотой окрестности точки $a$ и множества $E$, то $A\leq B$
	(\textbf{предельный переход в неравенствах});\\
	\textbf{6)} если существует такая
	$\stackrel{\circ}{U}_{\delta}(a),$ что
	$f(x)\leq h(x)\leq g(x)$ $\forall
		x\in E\cap\stackrel{\circ}{U}_{\delta}(a)$
	и $A=B,$ то
	$\lim\limits_{x\rightarrow a}h(x)=A$
	(\textbf{лемма о зажатом пределе});\\
	\textbf{7)} существуют такие $\delta>0$
	и $C\geq0,$ что
	$|f(x)|\leq C$ $\forall x\in$
	$E\cap\stackrel{\circ}{U}_{\delta}(a)$
	(\textbf{ограниченность функции, имеющей предел});\\
	\textbf{8)} если $A\neq0,$ то существует такая
	$\stackrel{\circ}{U}_{\delta}(a),$ что
	$|f(x)|\geq\frac{|A|}{2}$ $\forall x\in
		E\cap\stackrel{\circ}{U}_{\delta}(a)$
	(\textbf{лемма об отделимости}).
\end{theorem}

\newpage
\begin{problem}
Теорема о пределе композиции. Эквивалентные функции.
\end{problem}

\begin{theorem} \textbf{(Теорема о пределе композиции).}
	Пусть функция $g$ определена на множестве $D,$
	$b$ -- предельная точка множества $D$ и
	$$
		\lim\limits_{y\rightarrow b}g(y)=A\;(y\in D).
	$$
	Пусть функция $f: \ E\rightarrow D,$
	$a$ -- предельная точка множества $E$ и
	$\lim\limits_{x\rightarrow a}f(x)=b\;(x\in E).$
	Пусть, наконец, для некоторого $\delta>0$
	при всех $x$ из множества
	$E\cap\stackrel{\circ}{U}_{\delta}(a)$
	выполнено $f(x)\neq b.$
	Тогда сложная функция $g\circ f$ определена
	на множестве $E$ и
	$\lim\limits_{x\rightarrow a}g(f(x))=A.$
\end{theorem}

\begin{definition}
	Пусть функция $g$ определена и не равна нулю
	на множестве $E,$
	$f$ определена на множестве
	$E,$ $a$ -- предельная точка множества
	$E.$ Говорят, что функции $f$ и $g$
	\textbf{эквивалентны} при $x\rightarrow a,$
	если $\lim\limits_{x\rightarrow0}
		\frac{f(x)}{g(x)}=1.$ Обозначение:
	$f\sim g,\;x\rightarrow a.$
\end{definition}
\newpage
\begin{problem}
Критерий Коши существования предела функции.
\end{problem}

\begin{theorem} (\textbf{Критерий Коши.})
	Пусть функция $f$ определена на множестве $E,$
	$a$ -- предельная точка множества $E.$
	Функция $f$ имеет предел в точке $a$
	\textbf{тогда и только тогда}, когда для любого
	числа $\varepsilon>0$ существует такое число
	$\delta>0,$ что для любых чисел $x, y \in E,$
	удовлетворяющих неравенствам $0<|x-a|<\delta,$
	$0<|y-a|<\delta$ выполнено неравенство $|f(x)-
		f(y)|<\varepsilon.$ С помощью кванторов:
	$$\exists \lim\limits_{x\rightarrow a}f(x)\;
		\Leftrightarrow\;\forall \varepsilon > 0\;
		\exists \delta >0 :\;\forall \;x, y \in\;
		\stackrel{\circ}{U}_{\delta}(a) \cap E\;
		|f(x) - f(y)| < \varepsilon.
	$$
\end{theorem}

\newpage
\begin{problem}
Определение односторонних пределов. Определение монотонной функции. Определение ограниченной функции. Теорема Вейерштрасса для монотонной функции.
\end{problem}

\begin{definition}
	Пусть $a$ -- предельная точка множества $E^+_a.$
	Число $A$ называется \textbf{пределом справа}
	функции $f$ в точке $a,$ если
	$$
		\lim\limits_{E^+_a\ni x\to a}f(x)=a,
	$$
	то есть для любого
	$\varepsilon>0$ существует такое $\delta>0,$
	что при всех $x \in
		E^+_a\cap\stackrel{\circ}{U}_{\delta}(a)$
	$|f(x)-A|<\varepsilon.$
	Обозначение: $\lim\limits_{x\rightarrow a+0}f(x)=A$
	или $\lim\limits_{x\rightarrow a+}f(x)=A.$
	Аналогично определяется \textbf{предел слева} функции
	$f$ в точке $a,$ обозначаемый
	$\lim\limits_{x\rightarrow a-0}f(x)$
	или $\lim\limits_{x\rightarrow a-}f(x)=A$
	только множество $E^+_a$ в
	определении заменяется на $E^-_a.$
	Пределы справа и слева называются также
	\textbf{односторонними пределами.}
\end{definition}

\begin{definition}
	Если для любых таких $x_1, x_2 \in E,$
	что $x_1<x_2,$ выполнено неравенство
	$f(x_1)\leq f(x_2),$ то функция $f$
	называется \textbf{неубывающей} на множестве
	$E.$ Если выполнено неравенство
	$f(x_1)<f(x_2),$ то функция называется
	\textbf{возрастающей} на множестве $E.$ Если выполнены
	противоположные неравенства, то функция
	называется соответственно \textbf{невозрастающей}
	и \textbf{убывающей} на множестве $E.$
	Функция любого из четырёх
	указанных видов называется \textbf{монотонной}
	на множестве $E$ функцией.
\end{definition}

\begin{definition}
	Функция $f$
	называется ограниченной на множестве $M$,
	если она определена на этом множестве и
	существует такая константа $C>0,$
	что $|f(x)|\leq C$ при всех $x \in E.$
\end{definition}

\begin{theorem} (\textbf{Теорема Вейерштрасса}).
	1) Пусть функция $f$ определена на множестве $E$
	и $a$ -- предельная точка
	множества $E^-_a.$
	Пусть $f$ не убывает и ограничена сверху
	на множестве $E^-_a.$ Тогда существует предел слева
	функции $f$ в точке $a$ и имеет место равенство
	$
		\lim\limits_{x\rightarrow a-0}f(x)=
		\sup\limits_{x \in E^-_a}f(x).
	$\\
	2) Пусть функция $f$ не убывает и ограничена
	на множестве $E.$ Пусть $a$ -- предельная точка
	множества $E^+_a.$ Тогда существует предел справа
	функции $f$ в точке $a$ и имеет место равенство
	$
		\lim\limits_{x\rightarrow a+0}f(x)=
		\inf\limits_{x \in E^+_a}f(x).
	$
\end{theorem}

\newpage
\begin{problem}
Определение бесконечно малой функции.
\end{problem}

\begin{definition} Функция $f: E\rightarrow\R$
	называется бесконечно малой при
	$x\rightarrow a,$ где $a$ является предельной
	точкой множества $E,$ если
	$\lim\limits_{x\rightarrow a}f(x)=0.$
	Если
	$\lim\limits_{x\rightarrow a}f(x)
		=\infty$ (или $-\infty,$
	или $+\infty$), то функция
	$f(x)$ называется бесконечно большой
	при $x\rightarrow a.$ Везде $x \in E.$
\end{definition}


\newpage
\begin{problem}
Определение o-малого (Определение 5, Лекция 12). o-малое в терминах пределов
(Предложение 1, Лекция 12).
\end{problem}


\begin{definition} Пусть функции
	$f$ и $g$ определены на множестве
	$E,$ $a$ -- предельная точка множества
	$E.$ Говорят, что функция $f$ является
	бесконечно малой по сравнению с функцией
	$g$ при $x\rightarrow a,$ если
	$f(x)=h(x)g(x)$ и $h$ -- бесконечно
	малая функция при $x\rightarrow a.$
	Если при этом сами функции $f$ и $g$
	являются бесконечно малыми при
	$x\rightarrow a,$ то говорят, что
	функция $f$ -- \textbf{бесконечно
		малая более высокого порядка} по сравнению
	с $g$ при $x\rightarrow a.$
	Тот факт, что $f$ является бесконечно
	малой по сравнению с $g$ при $x\rightarrow a,$
	записывают в виде $f=o(g), \ x\rightarrow a$
	(читается ``$f$ равно о-малое от $g$ при
	$x,$ стремящемся к $a$''). Запись $f=o(1), \
		x\rightarrow a$ означает, что $f$ является
	бесконечно малой функцией при
	$x\rightarrow a.$ Таким образом,
	запись $f=o(g), \ x\rightarrow a$
	равносильна $f=o(1)g, \ x\rightarrow a.$
\end{definition}

\begin{lemma}
	Запись $f=o(g), \ x\rightarrow a$ равносильна
	также тому,
	что $\lim\limits_{x\rightarrow a}
		\frac{f(x)}{g(x)}=0$
	(при этом считаем, что $g(x)\neq0$
	в некоторой проколотой окрестности
	точки $a$).
\end{lemma}

\newpage
\begin{problem}
Асимптотические равенства 1 – 8 (Лекция 12). Асимптотические равенства 9 - 13.
\end{problem}

$$
	\sin x=x+o(x), \ x\rightarrow0. \eqno(1)
$$
$$
	\cos x=1-\frac{x^2}{2}+o(x^2), \ x\rightarrow0, \eqno(2)
$$
$$
	\tg x=x+o(x), \ x\rightarrow0, \eqno(3)
$$
$$
	\arcsin x=x+o(x), \ x\rightarrow0, \eqno(4)
$$
$$
	\arctg x=x+o(x), \ x\rightarrow0, \eqno(5)
$$
$$
	e^x=1+x+o(x), \ x\rightarrow0, \eqno(6)
$$
$$
	\ln(1+x)=x+o(x), \ x\rightarrow0, \eqno(7)
$$
$$
	(1+x)^{\alpha}=1+\alpha x+o(x), \ x\rightarrow0, \ \alpha\in\R. \eqno(8)
$$
$$
	e^x=1+\frac{x}{1!}+\frac{x^2}{2!}+\frac{x^3}{3!}+...
	+\frac{x^n}{n!}+o(x^n)=\sum\limits_{k=0}\limits^{n}
	\frac{x^k}{k!}+o(x^n), \ x\rightarrow0; \eqno(9)
$$
$$
	\sin x=\frac{x}{1!}-\frac{x^3}{3!}+\frac{x^5}{5!}-...
	+\frac{(-1)^{n}}{(2n+1)!}x^{2n+1}+o(x^{2n+1})=
	\sum\limits_{k=0}\limits^{n}
	\frac{(-1)^k}{(2k+1)!}x^{2k+1}+o(x^{2n+2}), \ x\rightarrow0; \eqno(10)
$$
$$
	\cos x=1-\frac{x^2}{2!}+\frac{x^4}{4!}-\frac{x^6}{6!}+...
	+\frac{(-1)^{n}}{(2n)!}x^{2n}+o(x^{2n})=
	\sum\limits_{k=0}\limits^{n}
	\frac{(-1)^k}{(2k)!}x^{2k}+o(x^{2n+1}), \ x\rightarrow0; \eqno(11)
$$
$$
	\ln (1+x)=x-\frac{x^2}{2}+\frac{x^3}{3}-\frac{x^4}{4}+...
	+\frac{(-1)^{n-1}}{n}x^{n}+o(x^{n})=
	\sum\limits_{k=1}\limits^{n}
	\frac{(-1)^{k-1}}{k}x^{k}+o(x^{n}), \ x\rightarrow0; \eqno(12)
$$
$$
	(1+x)^{\alpha}=1+\frac{\alpha}{1!}x+\frac{\alpha(\alpha-1)}{2!}x^2
	+\frac{\alpha(\alpha-1)(\alpha-2)}{3!}x^3+...+
$$
$$
	+\frac{\alpha(\alpha-1)(\alpha-2)\cdot...\cdot(\alpha-n+1)}{n!}
	x^{n}+o(x^{n})=
$$
$$
	=\sum\limits_{k=0}\limits^{n}
	\frac{\alpha(\alpha-1)(\alpha-2)\cdot...\cdot(\alpha-k+1)}{k!}
	x^{k}+o(x^{n}), \ x\rightarrow0 \ (\alpha\in\R).   \eqno(13)
$$