\begin{problem}Шесть определений непрерывной функции.
\end{problem}
\begin{problem}Определение непрерывности в точке справа и слева. Критерий непрерывности в терминах непрерывности слева и справа (Предложение 1, Лекция 13).
\end{problem}
\begin{problem}Локальные свойства непрерывных функций.
\end{problem}
\begin{problem}Теорема о композиции непрерывных функций. Точка разрыва.
\end{problem}
\begin{problem}Устранимый разрыв, разрыв первого рода и разрыв второго рода. Разрывы монотонной
на интервале функции. Определение непрерывной на множестве функции.
\end{problem}
\begin{problem}Теорема о нуле непрерывной на отрезке функции (Теорема 1, Лекция 13). Определение
ограниченной на множестве функции. 1-я теорема Вейерштрасса.
\end{problem}
\begin{problem}2-я теорема Вейерштрасса Теорема Больцано – Коши о промежуточном значении.
\end{problem}
\begin{problem}Определение равномерной непрерывности. Теорема Гейне – Кантора.
\end{problem}
\begin{problem}Обратная функция. Критерий непрерывности монотонной функции. Теорема об обратной функции.
\end{problem}
\begin{problem}Определение дифференцируемой функции. Определение дифференциала. Дифференциал как линейная функция (Лекция 16).
\end{problem}
\begin{problem}Определение производной. Связь дифференцируемости и производной (Предложение
1, Лекция 16). Определение касательной.
\end{problem}
\begin{problem}Непрерывность дифференцируемой функции. Определение равномерной сходимости.
\end{problem}
\begin{problem}Формулировка теоремы о равномерной сходимости последовательности непрерывных
функций. Пример Вейерштрасса непрерывной, но недифференцируемой функции.
\end{problem}
\begin{problem}Формулировка правил дифференцирования. Теорема о производной сложной функции.
\end{problem}
\begin{problem}Инвариантность формы первого дифференциала. Теорема о производной обратной
функции.
\end{problem}
\begin{problem}Таблица производных.
\end{problem}
\begin{problem}Определения локальных минимума и максимума и локального экстремума. Теорема
Ферма.
\end{problem}
\begin{problem}Теорема Ролля. Теорема Лагранжа. Геометрические смыслы этих теорем.
\end{problem}
\begin{problem}Два следствия теоремы Лагранжа. Теорема Коши.\end{problem}