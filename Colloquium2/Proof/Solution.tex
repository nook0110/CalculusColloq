\begin{problem}
Критерий непрерывности в терминах односторонних пределов (Предложение 1, Лекция
13). Доказательство локальных свойств непрерывных функций. Доказательство теоремы
о композиции непрерывных функций.
\end{problem}

\newpage
\begin{problem}
Примеры устранимых разрывов и разрывов 1 и 2 рода. Пример функции, разрывной в
каждой точке. Разрывы функции, монотонной на интервале (Предложения 4, Лекция 13).
\end{problem}

\newpage
\begin{problem}
Доказательство теоремы о нуле непрерывной на отрезке функции (Теорема 1, Лекция
13). Доказательство 1-й теоремы Вейерштрасса.
\end{problem}

\newpage
\begin{problem}
Доказательство 2-й теоремы Вейерштрасса. Примеры, демонстрирующие важность условий про непрерывность и отрезок в теоремах Вейерштрасса. Доказательство теоремы
Больцано – Коши о промежуточном значении.
\end{problem}
\newpage

\begin{problem}
Примеры равномерно непрерывных и неравномерно непрерывных функций (Лекция
14). Доказательство теоремы Гейне – Кантора (любое).
\end{problem}

\begin{example}
    1) Функция $f(x)=\sin x, x \in \R$
    равномерно непрерывна на $\R,$ т. к.
    $\forall \varepsilon>0$
    имеем: $|\sin x-\sin y|=
        |2\sin\frac{x-y}{2}\cos\frac{x+y}{2}|
        \leq2|\sin\frac{x-y}{2}|\leq|x-y|<\varepsilon,$
    если $|x-y|<\delta,$ где $\delta=
        \varepsilon.$ Таким образом, здесь выбор
    $\delta$ не зависит от точек на вещественной
    оси, а зависит только от $\varepsilon.$\\
    2) Функция $f(x)=\sin\frac{1}{x}, x > 0$
    не является равномерно непрерывной на $\R_+,$
    так как точки вида $x=\frac{1}{\frac{\pi}{2}+
            2\pi n}$ и $y=\frac{1}{-\frac{\pi}{2}+2\pi n}$
    ($n \in \N$) могут быть сколь угодно близки
    друг к другу при достаточно больших $n,$
    но $|\sin\frac{1}{x}-\sin\frac{1}{y}|=2$
    при всех натуральных $n,$ то есть при
    $0<\varepsilon<2$ необходимо будет подбирать
    $\delta$ в зависимости от конкретной точки.
    Рисунок 3 ниже демонстрирует это.
    Красная часть графика пересекает верхнюю
    и нижнюю стороны одного из
    прямоугольников. Чтобы этого не было,
    нужно уменьшить длину $\delta$ горизонтальной
    стороны при приближении к нулю. Это значит,
    что равномерной непрерывности нет.\begin{figure}[h!]
        \center{\includegraphics[scale=0.5]{sin1x.png}}
        \caption{Чем ближе к нулю, тем быстрее растёт функция на промежутках монотонности.}
        \label{fig:image}
    \end{figure}
\end{example}

\begin{theorem}(\textbf{Гейне -- Кантора о равномерной
        непрерывности}).
    Функция $f,$ непрерывная на отрезке $[a, b],$
    равномерно непрерывна на этом отрезке.
\end{theorem}
\begin{proof} \textbf{(1-й способ, через
        принцип Бореля -- Лебега).}
    По условию функция $f$ непрерывна в каждой точке
    отрезка $[a, b],$ поэтому в силу критерия Коши
    для функций при всяком $\varepsilon>0$
    для каждой точки $x\in[a, b]$
    найдётся такая окрестность $U_{\delta}(x),$
    что при всех $x, y\in\;U_{\delta}(x)$
    выполнено неравенство $|f(x)-f(y)|<\varepsilon.$
    Отметим, что $\delta$ зависит от точки
    $x,$ то есть для разных точек найдутся,
    вообще говоря, разные $\delta.$
    Все окрестности вида $U_{\delta/2}(x)$покрывают
    отрезок $[a, b],$ так как каждая
    точка отрезка принадлежит
    какой-нибудь из этих окрестностей.

    По принципу Бореля -- Лебега из
    системы всех окрестностей можно
    выбрать конечную систему,
    также покрывающую отрезок $[a, b].$
    Обозначим эти окрестности
    $$
        U_{\delta_i/2}(x_i),
        \textrm{ где } i=1,\;2,\;...,\;n.
    $$
    Положим
    $\delta=\min\{\frac{\delta_1}{2},\;
        \frac{\delta_2}{2},\;...,\;\frac{\delta_n}{2}\}$
    и докажем, что для любых таких точек $x',\;
        x''\in\;[a, b],$ что $|x'-x''|<\delta,$
    справедливо неравенство $|f(x')-f(x'')|
        <\varepsilon.$ Действительно, так как
    окрестности $\{U_{\delta_i/2}
        (x_i)\}_{i=1}^n$ покрывают
    отрезок $[a, b],$ то найдётся
    такая окрестность
    $$
        U_{\delta_j/2(x_j)},\;
        \textrm{ что }
        x'\in U_{\delta_j/2}.
    $$
    Тогда
    $$
        |x''-x_j|\leq|x'-x_j|+|x''-x'|<\delta_j/2+
        \delta\leq\delta_j,
    $$
    то есть $x''\in\;U_{\delta_j}(x_j),$
    поэтому, вспоминая, как были построены
    все окрестности вида $U_{\delta}(x),$
    мы имеем $|f(x')-f(x'')|<\varepsilon.$
    Таким образом, равномерная непрерывность
    доказана.
\end{proof}
\begin{proof} \textbf{(2-й способ, через
        теорему Больцано -- Вейерштрасса).}
    Предположим, что функция $f$ не является равномерно
    непрерывной на отрезке $[a, b].$
    Тогда существует такое $\varepsilon_1>0,$ что для любого
    натурального $n$ найдутся такие точки
    $x_n\in[a, b]$ и $y_n\in[a, b],$ что $|x_n-y_n|
        <\frac{1}{n},$ но
    $$
        |f(x_n)-f(y_n)|\geq\varepsilon_1. \eqno(1)
    $$
    Все элементы последовательностей $\{x_n\}_{n=1}^{\infty}$
    и $\{y_n\}_{n=1}^{\infty}$ принадлежат
    отрезку $[a, b],$ поэтому в силу теоремы Больцано --
    Вейерштрасса из них можно выбрать
    сходящиеся подпоследовательности
    $\{x_{n_k}\}_{k=1}^{\infty}$ и
    $\{y_{n_k}\}_{k=1}^{\infty}.$
    Так как при $n\rightarrow\infty$
    $|x_n-y_n|\rightarrow0,$ то пределы
    этих подпоследовательностей при $k\rightarrow
        \infty$ совпадают. Пусть эти пределы
    равны $x_0.$ $x_0\in[a, b]$ по предельному
    переходу в неравенствах, а тогда по условию
    функция $f$ непрерывна в точке $x_0.$
    В силу определения по Гейне непрерывной функции
    при любом $\varepsilon>0$ $|f(x_{n_k})-f(x_0)|
        <\frac{\varepsilon}{2}$ и $|f(y_{n_k})-f(x_0)|
        <\frac{\varepsilon}{2}.$ Если взять
    $\varepsilon=\varepsilon_1,$ то
    получим цепочку неравенств:
    $$
        \varepsilon_1 > |f(x_{n_k})-f(x_0)|+
        |f(y_{n_k})-f(x_0)| \geq |f(x_{n_k})-f(y_{n_k})|
        \geq \varepsilon_1.
    $$
    Таким образом, приходим к противоречию
    с неравенством (1).
\end{proof}

\newpage
\begin{problem}
Доказательство критерия непрерывности монотонной функции. Доказательство теоремы об обратной функции.
\end{problem}

\begin{theorem}(\textbf{Критерий непрерывности монотонной
        функции}).
    Монотонная на отрезке $[a, b]$ функция $f,$ непрерывна
    на  этом отрезке тогда и только тогда,
    когда множеством её значений является отрезок
    с концами $f(a)$ и $f(b)$.
\end{theorem}
Перед доказательством отметим, что $f(a)\leq f(b),$
если $f$ не убывает, и $f(a)\geq f(b),$ если
$f$ не возрастает.
\begin{proof}
    \textbf{Необходимость.} Если $c \in [a, b],$
    то $f(c)$ лежит между $f(a)$ и $f(b)$ в силу
    монотонности. По теореме Коши о непрерывной
    функции функция $f$ принимает все промежуточные
    значения между $f(a)$ и $f(b).$ Тем самым доказано,
    что область значения функции $f$ -- это отрезок
    с концами $f(a)$ и $f(b)$.

    \textbf{Достаточность.} Предположим, что функция $f$
    монотонна на отрезке $[a, b],$ область
    её значений -- отрезок с концами $f(a)$ и $f(b),$
    но $f$ имеет разрыв в точке $x_0\in[a, b]$.
    Тогда по теореме о разрывах монотонной
    функции (см. предыдущую лекцию) у
    этой $f$ может быть разрыв только первого
    рода, то есть если $x_0\in(a, b),$ то в точке
    разрыва существуют левый
    и правый пределы, но они не равны между собой:
    $$
        \lim\limits_{x\rightarrow x_0-0}f(x)=A\neq B=
        \lim\limits_{x\rightarrow x_0+0}f(x).
    $$
    Один из интервалов $(A, f(x_0))$ или
    $(f(x_0), B)$ непуст, и в нём нет значений
    функции $f.$ В силу монотонности функции $f$
    этот интервал содержится в отрезке с концами
    $f(a)$ и $f(b),$ поэтому этот отрезок
    не входит целиком в область значений функции $f$.
    Если же $x_0=a,$ то в этой точке существует
    правый предел, равный $B\neq f(a),$
    а тогда интервал $(f(a), B)$ не содержит
    ни одного значения функции $f.$
    Случай $x_0=b,$ разбирается аналогично,
    только теперь в точке $x_0$ существует левый
    предел $A\neq f(b).$ В любом из
    случаев получаем противоречие с тем,
    что область значений функции
    является отрезком.
\end{proof}

\begin{theorem}(\textbf{Теорема об обратной
        функции.})
    Пусть функция $f$ непрерывна и строго монотонна
    (то есть возрастает или
    убывает) на отрезке $[a, b].$ Тогда функция $f$
    имеет обратную функцию $f^{-1}$, определенную
    на отрезке с концами $f(a)$ и $f(b),$ причём
    $f^{-1}$ строго монотонна и непрерывна на отрезке
    с концами $f(a)$ и $f(b)$ и характер
    монотонности функций $f$ и $f^{-1}$
    одинаковый.
\end{theorem}
\begin{proof}
    То, что образом функции $f$ является
    отрезок с концами $f(a)$ и $f(b),$
    сразу следует из предыдущей теоремы,
    то есть $f$ является сюръекцией
    отрезка $[a, b]$ на отрезок
    с концами $f(a)$ и $f(b).$
    Инъекция вытекает из строгой
    монотонности: двум разным значениям
    функции соответствуют два разных значения
    аргумента. Таким образом, $f$ является
    биекцией, поэтому обратная функция
    $f^{-1}$ существует по определению.

    Монотонность функции $f^{-1}$ следует из
    того, что если, например, $f$ возрастает,
    то $f^{-1}(f(x_1))<
        f^{-1}(f(x_2))\Leftrightarrow x_1<x_2,$
    то есть $f^{-1}$ тоже возрастает, и аналогично
    для убывания. По определению,
    функция $f^{-1}$ определена на отрезке
    с концами $f(a)$ и $f(b),$ а областью
    её значений является отрезок $[a, b],$
    поэтому функция $f^{-1}$ непрерывна в силу
    предыдущей теоремы.
\end{proof}
\newpage

\begin{problem}
Доказательство непрерывности и монотонности функции $f(x) = x^n, n \in \N$.
Построение
иррациональной степени числа e.
\end{problem}


\begin{center}
    \textsf{Непрерывность степенных функций и рациональные степени
        положительных чисел}
\end{center}

Прежде всего, отметим, что функция $f(x)=x$ непрерывна на всей оси, так как
$$
    \forall a\in\R\;\lim\limits_{x\rightarrow a}x=a=f(a),
$$
поэтому функция $f(x)=x^n=
    \underbrace{x \cdot x \cdot ... \cdot x}_n\; (n\in\N)$ непрерывна на всей оси
как произведение $n$ непрерывных функций. Отсюда следует непрерывность
многочлена на всей оси, так как любой многочлен представляет собой
линейную комбинацию непрерывных функций. Кроме того, любая рациональная
функция непрерывна во всех точках, где знаменатель не обращается в ноль,
так как она представляет собой отношение двух непрерывных функций.

Пусть теперь $x\geq 0.$
Докажем, что $f(x)=x^n$ строго возрастает.
Действительно, пусть $a>b>0,$ тогда
$$
    f(a)=a^n=a^{n-1} \cdot a > a^{n-1}
    \cdot b > a^{n-2} \cdot b^2 > ... > b^n=f(b),
$$
а $f(0)<f(a)\;\forall a>0.$

Из непрерывности и строгой
монотонности по теореме об обратной функции следует,
что для функции $f(x)=x^n$ существует обратная к $f$
функция $g(x) = \sqrt[n]{x}.$ При этом в силу той же
теоремы об обратной функции $g$ непрерывна и возрастает
при всех $x\geq0.$

Определим \textbf{степень с рациональным показателем}.

Пусть $x>0, \ y = \sqrt[n]x, \ z = \sqrt[n]{x^m}$.
Тогда
$y^n = x, \ y^{nm} = x^m, \ z^n = x^m,$
откуда $(y^m)^n = z^n$ и
$y^m = z \iff (\sqrt[n]x)^m = \sqrt[n]{x^m}.$
Таким образом, операции взятия корня
и возведения в в степень с целочисленным
показателем перестановочны и мы можем использовать
следующие обозначения:
$z = x^{m/n}, \ z^{-1} = x^{-m/n}.$

Покажем, что свойства, справедливые
для степеней с целыми показателями, выполняются
и для степеней с рациональными показателями.

Пусть $r = \frac{a}{b}, \ r_1 = \frac{a_1}{b_1},
    \ a, a_1 \in \mathbb Z, \ b, b_1 \in \mathbb N$.
Пусть $d = x^{\frac{1}{bb_1}}$. Тогда
$$
    x^r \cdot x^{r_1} = d^{ab_1} d^{a_1b} =
    d^{ab_1+a_1b} = x^{(ab_1+a_1b)/bb_1} = x^{r+r_1}
$$
и
$$
    (x^r)^{r_1} = (d^{ab_1})^{a_1/b_1} = d^{aa_1}
    = x^{aa_1/bb_1} = x^{rr_1}.
$$

Таким образом, мы можем определить
степень с рациональным показателем для любого
положительного числа. Покажем, что при
$x>1$ рациональная степень $x$ возрастает
с ростом показателя.

Действительно, при $x>1, \ r>r_1$ имеем
$d>1, \ ab_1>a_1b, \ d^{ab_1}> d^{a_1b}$
поэтому $x^r > x^{r_1}$, то есть при
возрастании $r \in \mathbb Q$ и $x>1$ $x^r$
тоже возрастает.
\begin{center}
    \textsf{Построение и основные свойства экспоненты}
\end{center}

Пусть теперь $x = e$. В предыдущем
пункте мы определили степень с рациональным
показателем, в частности, при любом основании
$x>1,$ поэтому число $e$ в любой рациональной
степени определено.

Наша ближайшая цель -- определить $e^\alpha,$
где $\alpha\notin\Q.$ Для этого нам потребуется
вспомогательное утверждение.
\begin{proposition}
    Пусть $r\in\Q$ и $0 < r < 1.$ Тогда
    $e^r <1+\frac{r}{1-r}.$
\end{proposition}
\begin{proof}
    Ранее при всех $b\in \N$ мы установили справедливость неравенств
    $$
        \left(1+\frac{1}{b}\right)^b<e<
        \left(1+\frac{1}{b}\right)^{b+1},
    $$
    а тогда
    $
        e^{\frac{1}{b+1}} < 1 + \frac{1}{b} < e^{\frac{1}{b}},
    $
    поэтому
    $$
        e^{\frac{1}{b+1}} < 1 + \frac{1}{b} = \frac{1}
        {1 - \frac{1}{b+1}}, \ e^{\frac{1}{-(b+1)}} >
        1 - \frac{1}{b+1}.
    $$

    Положим $|r|<1, \ r = \frac{m}{n}.$
    Тогда $|m|<n$ и по неравенству Бернулли
    $$
        (e^{\pm \frac{1}{n}})^{|m|} \geq
        \left(1 \pm \frac{1}{n}\right)^{|m|} \geq
        1 \pm \frac{|m|}{n},
    $$
    поэтому $e^{m/n} = e^r \geq 1+r.$
    Тогда при
    $0 < r < 1$ выполнено $e^{-r}>1-r,$ что
    равносильно
    $$
        \ e^r <
        \frac{1}{1-r} = 1+\frac{r}{1-r}.
    $$
\end{proof}

Теперь определим $e^\alpha,$ где $\alpha \in
    \mathbb R\setminus \mathbb Q$. Возьмём такие рациональные
числа $r_1, r_2,$ что $r_1 < \alpha < r_2.$
Так как с ростом рационального показателя $r$
степень $e^r$ тоже возрастает, то при всех
числах $r_1 : r_1 \in \mathbb Q \wedge r_1
    < \alpha$ множество $M_1 = \{e^{r_1}| r_1
    \in  \mathbb Q
    \wedge r_1 < \alpha\}$
будет ограничено сверху числом $e^{r_2}$
поэтому существует
$\sup\limits_{r_1<\alpha} \{M_1\} = \gamma_1.$
Точно так же доказывается, что существует
$\gamma_2 = \inf\limits_{r_2>\alpha} M_2$,
где $M_2 = \{e^{r_2}| r_2 \in  \mathbb Q
    \wedge r_2 > \alpha\}$.

Покажем, что $\gamma_1 = \gamma_2$.
Так как каждое число $e^{r_2}$ является верхней
гранью множества $M_1$, а $\gamma_1 = \sup M_1$,
то $\gamma_1 \leq e^{r_2}$, причем это
верно для всех $r_2>\alpha,$ поэтому
$\gamma_1$ -- нижняя грань для $M_2,$
а тогда $\gamma_1 \leq \gamma_2 = \inf M_2$.

Выберем такие рациональные $r_1$ и $r_2,$ что
$$
    [\alpha] < r_1 < \alpha < r_2 < [\alpha] + 1.
$$
Таким образом, справедливы неравенства
$$
    e^{r_1} \leq \gamma_1 \leq \gamma_2
    \leq e^{r_2} \leq e^{[\alpha] + 1}
$$
поэтому
$$
    0 \leq \gamma_2 - \gamma_1 \leq e^{r_2} - e^{r_1} =
    e^{r_1}(e^{r_2 - r_1} - 1) \leq e^{[\alpha] +1}
    \frac{r_2 - r_1}{1 - (r_2 - r_1)},
$$
где в последнем переходе использовано
предложение 1.

При этом $\gamma_2 - \gamma_1$ -- фиксированное число,
а разность $r_2-r_1$ положительна
и может быть сколь угодно малой, так как в любой
окрестности $\alpha$ есть рациональные числа, а тогда
$\gamma_2 = \gamma_1$.
Положим $\gamma_2 = \gamma_1 = e^\alpha$.
Таким образом, мы определили функцию
$y = e^x$ при всех $x\in\R$.

\newpage

\begin{problem}
Доказательство монотонности экспоненты и равенства $e^{x+y}=e^xe^y$. Доказательство
непрерывности показательной функции на $\R$.
\end{problem}


Докажем теперь основные свойства этой
функции.
\begin{proposition}
    1) $f(x) = e^x$ возрастает на $\mathbb R;$
    2) $\forall \alpha, \beta \in \mathbb R
        \ e^{\alpha+\beta} = e^{\alpha}e^{\beta}$.
\end{proposition}
\begin{proof}
    1) Если $r_1 < \alpha < r_2, \ r_1, r_2 \in \mathbb Q,$
    то
    $$
        e^{r_1} < e^\alpha < e^{r_2}
    $$
    в силу
    определения степени $e$ с иррациональным
    показателем. Если теперь
    $\alpha < \beta$, то существует такое $r_3 \in
        \mathbb Q \cup(\alpha, \beta),$
    что
    $$
        e^\alpha < e^{r_3} < e^{\beta}
    $$
    откуда
    получаем, что $f(x)=e^x$ строго монотонна.

    2) Для рациональных
    чисел это свойство доказано
    в предыдущем разделе.

    Пусть $\mu = \alpha + \beta$.
    Если $\mu \in \mathbb R\setminus\Q$, то
    $$
        e^\mu =
        \sup\limits_{r_1<\mu} e^{r_1} =
        \inf\limits_{r_2>\mu}e^{r_2}.
    $$
    Пусть $r_1 = r'_1+r''_1$,
    где $r'_1<\alpha, \ r''_1<\beta, \ r_2 =
        r'_2+r''_2, \ r'_2 > \alpha, \ r''_2>\beta.$
    Тогда
    $$
        e^{r_1} = e^{r'_1+r''_1}<e^\alpha
        e^\beta < e^{r'_2+r''_2} = e^{r_2}.
    $$
    Но при этом
    $\ e^{r_1} < e^\mu < e^{r_2},$ откуда получаем
    $$
        |e^\mu - e^\alpha e^\beta| < e^{r_2} - e^{r_1}.
    $$
    Разность $|e^\mu - e^\alpha e^\beta|$
    может быть сколь угодно мала при произвольных
    рациональных $r_1, \ r_2$ с условием
    $r_1<\mu<r_2, \ r_1, \ r_2 \in \Q$
    (снова в силу предложения 1),
    поэтому
    $$
        e^\mu = e^{\alpha+\beta}=e^\alpha e^\beta.
    $$
\end{proof}
\begin{center}
    \textsf{Непрерывность показательной и тригонометрической функций}
\end{center}

В этом разделе мы докажем непрерывность
функции $f(x)=a^x, \ a>1,$ а тогда
непрерывность показательной функции легко
выводится и при $0<a<1,$ так как
$a=\frac{1}{b},$ где $b>1.$

Отметим, что определить показательную
функции с основанием $a$ можно,
используя уже определённую функцию $y=e^x$
и обратную к ней, то есть натуральный
логарифм, существование и свойства
которого после доказательства непрерывности
ниже и уже доказанных выше свойств экспоненты
станут очевидными.

В следующем предложении мы докажем непрерывность
показательной функции в каждой точке.
\begin{proposition}.
    В любой точке
    $x_0 \in \mathbb R$ функция $f(x)=a^x$ непрерывна.
\end{proposition}
\begin{proof}
    Пусть $a > 1$. Нам необходимо доказать, что
    $$
        \forall \varepsilon>0 \ \exists \delta>0:
        \forall x: |x-x_0|<\delta \ |a ^x - a^{x_0}| <
        \varepsilon
    $$
    Это равносильно тому, что $|a^{x - x_0} - 1|
        < \varepsilon \cdot a^{-x_0} = \varepsilon_1$.
    Мы сразу можем считать, что $\varepsilon_1 < 1$.
    Итак, в качестве $\delta$ мы хотим подобрать
    такое число $\delta_1,$ что
    при $|x - x_0|< \delta_1 \ |a^{x - x_0}-1|<\varepsilon_1.$
    %Пусть $\delta_1 =\frac{\varepsilon_1}{a+1} > 0.$

    Так как $-\delta_1 < x - x_0 < \delta_1$
    и $a>1,$ то
    $$
        a^{-\delta_1} < a^{x - x_0} < a^{\delta_1},
    $$
    что равносильно
    $$
        a^{-\delta_1} - 1 < a^{x - x_0} - 1 < a^{\delta_1} - 1.
    $$
    Подберём $\delta_1$ так, чтобы
    выполнялось неравенство
    $a^{\delta_1} - 1 < \varepsilon_1.$
    Отметим, что тогда
    $$
        a^{-\delta_1}>
        \frac{1}{1+\varepsilon_1} = 1 - \frac{\varepsilon_1}
        {1 + \varepsilon_1} > 1 - \varepsilon_1,
    $$
    то есть будет выполнено неравенство
    $|a^{x - x_0}-1|<\varepsilon_1.$

    Пусть $N = \left[\frac{1}{\delta_1}\right],$
    тогда $\frac{1}{\delta_1} \geq N,$ откуда
    следует $a^{1/N} \geq a^{\delta_1},$
    поэтому если будет выполнено неравенство
    $1 + \varepsilon_1>a^{1/N},$ то
    тем более будет верно
    $a^{\delta_1} - 1 < \varepsilon_1.$

    Используя неравенство
    Бернулли, можем подобрать $N$ (а тогда и $\delta_1$)
    из неравенства
    $$
        (1 + \varepsilon_1)^N \geq 1 + N\varepsilon_1>a.
    $$
    Если $N>\frac{a}{\varepsilon_1},$ то
    $1 + N\varepsilon_1 > 1 +
        \frac{a}{\varepsilon_1}\varepsilon_1 > a,$
    то есть неравенство
    заведомо выполнено. Тогда
    в силу задания $N$ можно положить
    $\delta_1 =\frac{\varepsilon_1}{a+1},$
    так как в этом случае получим
    $$
        N = \left[\frac{1}{\delta_1}\right] = \left[\frac{a+1}{\varepsilon_1}\right]
        \geq \left[\frac{a + \varepsilon_1}{\varepsilon_1}\right] =
        \left[\frac{a}{\varepsilon_1}\right]+1 > \frac{a}{\varepsilon_1}.
    $$

    Таким образом, подобрано
    $\delta_1,$ при котором
    $$ -\varepsilon_1 < a^{-\delta_1} - 1
        < a^{x - x_0} - 1 < a^{\delta_1} - 1
        < \varepsilon_1,
    $$
    чем и доказана непрерывность функции
    $f(x)=a^x$ в точке $x_0.$
\end{proof}


\newpage

\begin{problem}
Непрерывность функции $f(x) = \sin x$. Построение функций
\begin{equation}
    f(x) = \ln x, f(x) = \arctan x, f(x) = x^\alpha, \alpha \in \R.
\end{equation}
Доказательство свойств этих функций.
\end{problem}

Хотя мы и не будем строго определять
функцию $f(x)=\sin x,$ но докажем её непрерывность
при любом $x\in\R.$
\begin{proposition}.
    В любой точке
    $x_0 \in \mathbb R$ функция $f(x)=\sin x$ непрерывна.
\end{proposition}
\begin{proof}.
    При доказательстве первого замечательного предела
    для всех $x\in\R$ мы установили неравенство
    $|\sin x| \leq |x|.$
    Воспользовавшись им, получаем для фиксированного
    $x_0\in\R$
    $$
        |\sin x - \sin x_0| = 2|\sin\frac{x-x_0}{2}
        \cos\frac{x+x_0}{2}| \leq 2 |\frac{x - x_0}{2}|
        = |x-x_0|
    $$
    поэтому для любого
    $\varepsilon > 0$ можно взять
    $\delta = \varepsilon$ и тогда
    $$
        |\sin x - \sin x_0| < \varepsilon \ \forall x:
        |x-x_0| < \varepsilon,
    $$
    что и означает, что
    $\sin \in C(x_0)$.
\end{proof}


\begin{center}
    \textsf{Построение некоторых обратных функций}
\end{center}

Так как $f(x)=e^x$ монотонно возрастает и непрерывна
на всей прямой, то в силу теоремы об обратной
функции существует функция $g(x)=f^{-1}(x),$
отображающая луч $(0; +\infty)$ на $\mathbb R$.
Хотя теорему об обратной функции
мы доказали для отрезка, здесь обратная
функция существует именно на множестве
всех положительных чисел (почему?).
Эту называют натуральным логарифмом
$g(x) := \ln x, \ x>0$. По теореме об обратной
функции она непрерывна, возрастает и
$x = e^{\ln x}$ откуда следует
$$
    e^{\ln xy} = xy = e^{\ln x} e^{\ln y} = e^{\ln x+\ln y}
$$
то есть $\ln xy = \ln x + \ln y.$
Если $\alpha$ -- иррациональное число и $x>0,$
то $x^\alpha = e^{\alpha \ln x}.$
Таким образом, все свойства степенной
функции следует из доказанных
выше свойств показательной и логарифмической функций.

Так как $f(x)=\sin x$ непрерывна на отрезке
$[-\frac{\pi}{2}, \frac{\pi}{2}]$ и
монотонно возрастает на этом отрезке,
то можем определить обратную
функцию $g(x) = \arcsin x$. Она определена при
$-1 \leq x \leq 1$, отображает этот отрезок в
$[-\frac{\pi}{2}, \frac{\pi}{2}]$ и монотонно возрастает.

Если $f(x)=\tg x,$ то $f \in C(-\frac{\pi}{2},
    \frac{\pi}{2})$ и возрастает на этом интервале,
$f(x)=\tg x \in \R$ поэтому $g(x)=\arctg x$ определена
при всех $x \in \R$, $g(x)=\arctg x \in (-\frac{\pi}{2},
    \frac{\pi}{2})$ и монотонно возрастает.

Аналогично можно определить $g(x)=\arccos x$ и
$g(x)=\arcctg x.$

\newpage

\begin{problem}
Примеры вычисления дифференциала функции по определению. Доказательство равносильности дифференцируемости и наличия производной.
\end{problem}

\begin{example}
    1) Пусть $f(x)=x.$
    Тогда $f(a+h)-f(a)=(a+h)-a=h=1\cdot h+0\cdot h.$
    Из этого представления имеем: $A=1,
        \alpha(h)=0, dx(h)|_{x=a}=h.$\\
    2) Пусть $f(x)=x^2.$ Тогда
    $f(a+h)-f(a)=(a+h)^2-a^2=2ah+h^2=
        2ah+h\cdot h,$ то есть $A=2a, \ d(x^2)(h)|_{x=a}=
        2ah,$ $\alpha(h)=h.$\\
    3) Пусть $f(x)=x^3.$ Тогда
    $f(a+h)-f(a)=(a+h)^3-a^3=3a^2h+3h^2a+h^3=
        3a^2h+(3ah+h^2)h$ то есть $A=3a^2, \ d(x^3)(h)|_{x=a}=
        3a^2h,$ $\alpha(h)=3ah+h^2.$
\end{example}

\begin{proposition}
    Функция $f$ дифференцируема в точке $a$
    тогда и только тогда, когда в точке
    $a$ существует производная этой функции $f'(a).$
    При этом $df(h)|_{x=a}=f'(a)h.$
\end{proposition}
\begin{proof}
    \textbf{Необходимость.} Если функция $f$
    дифференцируема,
    то справедливо равенство
    $$
        f(a+h)-f(a)=Ah+
        \alpha(h)h, \
        \lim\limits_{h\rightarrow0}\alpha(h)=0.
    $$
    Разделив обе части этого
    равенства на $h,$ получим
    $$
        \frac{f(a+h)-f(a)}{h}=A+\alpha(h).
    $$
    Переходя к переделу при $h\rightarrow0$
    и учитывая, что $\lim\limits_{h\rightarrow0}
        \alpha(h)=0,$ получим, что $f'(a)=A.$\\
    \textbf{Достаточность.} Если функция $f$
    имеет производную, то есть существует
    предел
    $$
        \lim\limits_{h\rightarrow0}
        \frac{f(a+h)-f(a)}{h}=f'(a),
    $$
    то по определению
    предела справедливо равенство
    $$
        \frac{f(a+h)-f(a)}{h}=f'(a)+\alpha(h),
    $$
    где $\lim\limits_{h\rightarrow0}
        \alpha(h)=0.$
    Домножив обе части этого равенства
    на $h,$ мы получим равенство из
    определения дифференцируемой функции.
\end{proof}

\newpage

\begin{problem}
Уравнение касательной. Геометрическая интерпретация производной и дифференциала. Вывод производной $f(x) = e^x, f(x) = \cos x, f(x) = \sin x$.
\end{problem}


Изучим
\emph{геометрическую интерпретацию производной
    и дифференциала}.
Из равенства $h=x-a$ получим, что дифференцируемая
функция может быть записана в виде $f(x)=f(a)+
    f'(a)(x-a)+\alpha(x)(x-a),$ где $\lim\limits_{
        x\rightarrow a}\alpha(x)=0.$ Это значит, что в
некоторой окрестности точки $a$ функция $f$
приближается функцией $x\mapsto f(a)+f'(a)(x-a).$
Таким образом, локально (то есть в некоторой
окрестности точки $a$) график функции $f$
выглядит "почти" как прямая. Сама прямая
$y=f'(a)(x-a)+f(a)$ называется
\emph{касательной} к графику функции $f$ в точке
$a.$ Производная $f'(a)$
является тангенсом угла наклона касательной
к положительному направлению оси $Ox.$
На рисунке 1 красным цветом изображена
касательная. \begin{figure}[h!]
    \center{\includegraphics[scale=0.45]{Касательная.png}}
    \caption{Функция "сливается" с касательной.}
    \label{fig:image}
\end{figure} Обратим внимание, что
график функции и касательной неразличимы
в некоторой окрестности.
Приведём несколько примеров на вычисление
производных с помощью определения производной.
\begin{example}
    1) Пусть $f(x)=e^x.$
    Найдём $\frac{df}{dx}(a).$
    Имеем по определению:
    $$
        \frac{df}{dx}(a)=\lim\limits_{h\rightarrow0}
        \frac{e^{a+h}-e^a}{h}=e^a\lim\limits_{h\rightarrow0}
        \frac{e^{h}-1}{h}=e^a.
    $$
    Так как это верно в любой
    точке области определения,
    то при всех $x\in\R$
    $(e^x)'=e^x.$

    2) Пусть $f(x)=\cos x.$ Тогда
    $$
        f'(x)=\lim\limits_{h\rightarrow0}
        \frac{\cos(x+h)-\cos x}{h}=
        \lim\limits_{h\rightarrow0}
        \frac{-2\sin\frac{h}{2}\sin\frac{2x+h}{2}}{h}=
        -\lim\limits_{h\rightarrow0}
        \frac{\sin\frac{h}{2}\sin\frac{2x+h}{2}}
        {\frac{h}{2}}=-\sin x.
    $$
    Совершенно аналогично проверяется, что
    $\sin'(x)=\cos x.$\\
\end{example}

\newpage

\begin{problem}
Доказательство непрерывности дифференцируемой функции. Пример непрерывной и недифференцируемой в точке функции (с доказательством).
\end{problem}

\begin{proposition}
    Пусть функция $f$ дифференцируема
    в точке $a.$ Тогда $f$ непрерывна в точке $a.$
\end{proposition}
\begin{proof}
    Из определения дифференцируемости следует, что
    $$
        f(x)=f(a)+f'(a)(x-a)+o(x-a)\rightarrow f(a), \
        x\rightarrow a,
    $$
    то есть предел функции в точке $a$ равен
    её значению в этой точке, что по определению
    означает непрерывность.
\end{proof}
Следующий пример показывает, что обратное
неверно: непрерывная в точке функция может
быть недифференцируемой в этой точке.
Согласно предложению 1 это равносильно
тому, что такая функция в этой точке не имеет производную.
\begin{example}
    Функция $y=|x|$ недифференцируема в нуле.
    Действительно,
    $$
        \lim\limits_{h\to 0-}
        \frac{|h|}{h}=\lim\limits_{h\to 0-}
        \frac{-h}{h}=-1, \textrm{ а }
        \lim\limits_{h\to 0+}
        \frac{|h|}{h}=
        \lim\limits_{h\to 0-}
        \frac{h}{h}=1.
    $$
    Если бы производная
    в нуле существовала, то есть если бы
    существовал предел, то это было бы равносильно
    существованию и равенству односторонних пределов.
    Таким образом, у функции $y=|x|$
    не существует производной при $x=0,$ что равносильно
    тому, что эта функция недифференцируема в нуле.
\end{example}

\newpage

\begin{problem}
Доказательство правил дифференцирования. Запись этих правил с помощью дифференциалов. Вывод производных функций $f(x) = \tg x, f(x) = \sh x, f(x) = \ch x$.
\end{problem}
\newpage

\begin{problem}
Доказательство теоремы о производной сложной функции. Инвариантность формы
первого дифференциала.
\end{problem}
\newpage

\begin{problem}
Доказательство теоремы о производной обратной функции. Выводы формул таблицы
производных.
\end{problem}
\newpage

\begin{problem}
Доказательства теорем Ферма и Ролля. Примеры к невыполнению условий теоремы
Ролля.
\end{problem}
\newpage

\begin{problem}
Доказательства теоремы Лагранжа и её следствий.
\end{problem}
\newpage

\begin{problem}
Доказательство теоремы Коши и её физический смысл.
\end{problem}
\newpage
