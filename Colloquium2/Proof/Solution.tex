\begin{problem}
Критерий непрерывности в терминах односторонних пределов (Предложение 1, Лекция
13). Доказательство локальных свойств непрерывных функций. Доказательство теоремы
о композиции непрерывных функций.
\end{problem}
\begin{problem}
Примеры устранимых разрывов и разрывов 1 и 2 рода. Пример функции, разрывной в
каждой точке. Разрывы функции, монотонной на интервале (Предложения 4, Лекция 13).
\end{problem}
\begin{problem}
Доказательство теоремы о нуле непрерывной на отрезке функции (Теорема 1, Лекция
13). Доказательство 1-й теоремы Вейерштрасса.
\end{problem}
\begin{problem}
Доказательство 2-й теоремы Вейерштрасса. Примеры, демонстрирующие важность условий про непрерывность и отрезок в теоремах Вейерштрасса. Доказательство теоремы
Больцано – Коши о промежуточном значении.
\end{problem}
\begin{problem}
Примеры равномерно непрерывных и неравномерно непрерывных функций (Лекция
14). Доказательство теоремы Гейне – Кантора (любое).
\end{problem}
\begin{problem}
Доказательство критерия непрерывности монотонной функции. Доказательство теоремы об обратной функции.
\end{problem}
\begin{problem}
Доказательство непрерывности и монотонности функции $f(x) = x^n, n \in \N$.
Построение
иррациональной степени числа e.
\end{problem}
\begin{problem}
Доказательство монотонности экспоненты и равенства $e^{x+y}=e^xe^y$. Доказательство
непрерывности показательной функции на $\R$.
\end{problem}
\begin{problem}
Непрерывность функции $f(x) = \sin x$. Построение функций
\begin{equation}
    f(x) = \ln x, f(x) = \arctan x, f(x) = x^\alpha, \alpha \in \R.
\end{equation}
Доказательство свойств этих функций.
\end{problem}
\begin{problem}
Примеры вычисления дифференциала функции по определению. Доказательство равносильности дифференцируемости и наличия производной.
\end{problem}
\begin{problem}
Уравнение касательной. Геометрическая интерпретация производной и дифференциала. Вывод производной $f(x) = e^x, f(x) = \cos x, f(x) = \sin x$.
\end{problem}
\begin{problem}
Доказательство непрерывности дифференцируемой функции. Пример непрерывной и недифференцируемой в точке функции (с доказательством).
\end{problem}
\begin{problem}
Доказательство правил дифференцирования. Запись этих правил с помощью дифференциалов. Вывод производных функций $f(x) = \tg x, f(x) = \sh x, f(x) = \ch x$.
\end{problem}
\begin{problem}
Доказательство теоремы о производной сложной функции. Инвариантность формы
первого дифференциала.
\end{problem}
\begin{problem}
Доказательство теоремы о производной обратной функции. Выводы формул таблицы
производных.
\end{problem}
\begin{problem}
Доказательства теорем Ферма и Ролля. Примеры к невыполнению условий теоремы
Ролля.
\end{problem}
\begin{problem}
Доказательства теоремы Лагранжа и её следствий.
\end{problem}
\begin{problem}
Доказательство теоремы Коши и её физический смысл.
\end{problem}